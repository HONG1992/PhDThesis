
\tmpsection{What I did}
The overall aim of this thesis was to explain patterns of social trait evolution in the ants and to
quantify the effects of social traits on the patterns of diversification in the ants. I accomplished
this by using supertree methods to construct a complete genus-level phylogeny of the ants and
by creating a large database of social trait data. These tools enabled me to use rigorous
statistical comparative methods to identify the phenotype of the ancestral ant; to explore the
patterns of evolution on key social traits and test predictions of evolutionary associations
between these traits across the whole ant phylogeny; to test the predicted association between
colony size and division of labour (both non-reproductive and reproductive) on a finer scale
within the Attini; and to test hypotheses concerning the effects of social traits on diversification
patterns in the ants.

\begin{itemize}
\item In this thesis I aimed to test the importance of population structure and density on pathogen diversity
\item With a particular focus on bats
\item Combining simulations and empirical studies.
\item Identify that population structure does affect pathogen richness in wild bats.
\item But found that invasision of new pathogens is probably not the mechanism.  
\item Clarified the relationships between population size and density, range size, colony size and population structure.
\item And found that colony size is more important than density per se. 
\item Finally, created a method to more easily estimate bat population densities.
\end{itemize}



\tmpsection{How I did it (Chapters)}






\tmpsection{What was congruent with the literature?}  



\tmpsection{What was surprising?}

\begin{itemize}
\item Don't find high $R_0$ leads to high richness.
\item Find that colony size is more important that implying density is just proxy for group size.
\item 


\end{itemize}


\tmpsection{What are the limitations to the study?}



\tmpsection{What are the implications for practice?}


\begin{itemize}
\item Global change and pop structure.
\item Studies should more carefully consider density vs structure vs group size.
\item Can more easily estimate bat density.

\end{itemize}


\tmpsection{What are the implications for research?}



\tmpsection{Furtherwork}

\begin{itemize}
\item Examine more carefully the mechanisms for richness
\item Examine multi host species more carefully
\item Field test gREM
\end{itemize}


