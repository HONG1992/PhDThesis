\maketitle
\makedeclaration

\begin{abstract} % 300 word limit

\tmpsection{One or two sentences providing a basic introduction to the field}
% comprehensible to a scientist in any discipline.

\lettr{P}athogens make up a huge proportion of global diversity and their role in disease strongly affects human health, health and economic development as well as having important ecological roles.
However, the factors that control number of pathogen species are poorly understood.


\tmpsection{Two to three sentences of more detailed background}
% comprehensible to scientists in related disciplines.

% Theory led.
% 

The patterns of contacts between individuals --- in both human and animal populations --- are nonrandom and depend on the density of individuals.
Population structure and density have important epidemiological consequences, but their role in the control of pathogen richness is unknown.




\tmpsection{One sentence clearly stating the general problem (the gap)}
% being addressed by this particular study.





\tmpsection{One sentence summarising the main result}
%  (with the words “here we show” or their equivalent).

In this thesis I have studied how population structure and density control pathogen richness using bats as a case study.
I have used epidemiological simulations and comparative analyses of data from wild bat populations to show that population structure does not have a strong affect on pathogen richness.
Specifically, my empirical analyses of bat data from wild populations show opposite effects of population structure on pathogen richness depending on the measure used.
Further, simulations suggest that population structure does not alter the ability of new pathogens to invade and persist in bat populations.
Using further simulations I have shown that population density and population structure are interrelated.
Finally, I have created a model for estimating bat densities --- previously an incredibly challenging task --- using acoustic data.
Together, these studies clarify the relative roles of population density and structure and facilitate further study of population density in bats as a driver of pathogen richness.

\tmpsection{Two or three sentences explaining what the main result reveals in direct comparison to what was thought to be the case previously}
% or how the main result adds to previous knowledge

While theory previously predicted that population structure should increase pathogen richness, the expectation in the ecological literature was that population structure would decrease pathogen richness.
My studies support neither of these views, instead suggesting that population structure does not have a strong affect in either direction.
The role of bat density on pathogen richness has not been studied due to the difficulty in estimating population densities.
Previously it was often impossible to estimate  abundance from acoustic data; the model I have developed now allows density to be measured more easily, less invasively and across broader spatial and taxonomic scales.
It is hoped that once more estimates of bat density are acquired, the hypothesis that high density species carry more pathogens can be tested.

\tmpsection{One or two sentences to put the results into a more general context.}

Broadly, the factors that control pathogen diversity are still unclear, with studies being limited by data bias and sparsity.
My studies show that examining the same question using theoretical and empirical method provides scientific understanding that is more robust to these answer.
Further, I provide a tool to enable researchers to fill one of these gaps; the lack of density estimates for bats.


\tmpsection{Two or three sentences to provide a broader perspective, }
% readily comprehensible to a scientist in any discipline.




\end{abstract}

\begin{acknowledgements}
Acknowledge all the things!
\end{acknowledgements}

\setcounter{tocdepth}{2} 
% Setting this higher means you get contents entries for
%  more minor section headers.

\tableofcontents
\listoffigures
\listoftables

