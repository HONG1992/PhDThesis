\maketitle
%\makedeclaration

\begin{abstract} % 300 word limit

\lettr{P}athogens acquired from animals make up the majority of emerging human diseases, are often highly virulent and can have large affects on public health and economic development.
Identifying species with high pathogen species richness enables efficient sampling and monitoring of potentially dangerous pathogens.
I examine the role of host population structure and size in maintaining pathogen species richness in an important reservoir host for zoonotic viruses, bats (Order, Chiroptera). 
Firstly I test whether population structure is associated with high viral richness across bat species within a comparative, phylogenetic analysis. 
I find evidence that bat species with more structured populations have more virus species. 
As this type of study cannot distinguish between specific mechanisms, I then formulate epidemiological models to test whether more structured host populations may allow invading pathogens to avoid competition. 
However, these models show that increasing population structure decreases the rate of pathogen invasion. 
As both global host population structure and local group size appear to be important for disease invasion, I use the same modelling framework to compare the importance of host density, group size and number of groups. 
I find that host group size has a stronger affect than density or number of groups. 
There are few bat population size estimates to empirically test the importance of host population size on pathogen richness. 
Therefore, to assist future research, I develop a method for estimating bat population sizes from acoustic surveys. 
Overall in this thesis, I show that the structure and size of host bat populations can affect their ability to maintain many pathogen species and I provide a method to measure population sizes of bats. 
These findings increase our understanding of the ecological process of pathogen community construction and can help optimise surveillance for zoonotic pathogens.

\end{abstract}





%%%%%%%%%%%%%%%%%%%%%%%%%%%%%%%%%%%%%%%%%%%%%%%%%%%%%%%%%%%%%%
%% Acknowledgements                                         %%
%%%%%%%%%%%%%%%%%%%%%%%%%%%%%%%%%%%%%%%%%%%%%%%%%%%%%%%%%%%%%%


\begin{acknowledgements}


\tmpsection{Kat + dylan, mum and dad}

Firstly and most importantly I would like to thank my wife, Katrina, for helping me beyond measure throughout my PhD.
Although he has contributed very little towards my thesis, I would also like to thank my son, Dylan, for making my life tiring and brilliant for the last two years.
I would also like to thank my parents for the endless support they have given me both during and before my studies.


\tmpsection{supervisors}

I would like to give special thanks to my supervisors, Kate Jones and Hilde Wilkinson-Herbots.
Their guidance has been invaluable.
Their patience, as my drafts progressed from terrible to less-terrible, has been admirable.
Thank you for all your help.
A number of other collaborators and researchers have contributed to, or commented on, various chapters and analyses.
My thanks goes to David Redding, Marcus Rowcliffe, Robin Freeman, Chris Carbone, Andrew Cunningham, James Wood and Francois Balloux.

\tmpsection{Friends and that}

I have been fortunate enough to get to know a large number of fellow students during my time at UCL.
I would like to thank my CoMPLEX cohort for making the masters year hilarious.
In particular thank you Liz for being a great collaborator and for ensuring there was at least one other ecologist in CoMPLEX.
I would also like to thank all those at CBER for making the office such a social place to work: Dave Curnick, for constantly reminding me how poor at squash I am; Henry Ferguson-Gow, for the many music recommendations and discussions; as well as Allie Fairbrass, Ellie Dyer, Laura Nunes, Roee Moar, Prabu Sivasubramaniam, Gee Ferreira, Stuart Nattrass and everyone else.
It would have been very dull without you.


\end{acknowledgements}

\setcounter{tocdepth}{2} 
% Setting this higher means you get contents entries for
%  more minor section headers.

\tableofcontents
\listoffigures
\listoftables

