\maketitle
\makedeclaration

\begin{abstract} % 300 word limit

\tmpsection{One or two sentences providing a basic introduction to the field}
% comprehensible to a scientist in any discipline.
\lettr{I}t is still unclear what factors determine the number of pathogens a wild species carries.
But once understood, these factors could provide a way to prioritise surveillance of wild populations for zoonotic disease.


\tmpsection{Two to three sentences of more detailed background}
% comprehensible to scientists in related disciplines.

% Theory led.
% 

The pattern of contacts between individuals (i.e. population structure) has long been known to strongly affect epidemic processes.
Theory suggests that population structure can promote pathogen richness while the ecological literature generally assumes it will decrease richness.
It is still unclear how important population structure is in controlling pathogen richness in wild populations.
Previous studies have had contradictory results and the different measures of population structure have different shortcomings.


\tmpsection{One sentence clearly stating the general problem (the gap)}
% being addressed by this particular study.

Here I use comparative data to test whether population structure influences pathogen richness in bats.
I use two measures of population structure: a novel measure, number of subspecies, and a more careful application of genetic measures which have been used previously.

\tmpsection{One sentence summarising the main result}
%  (with the words “here we show” or their equivalent).

I find conflicting evidence suggesting that while population structure may promote pathogen diversity it is likely not a strong affect.


\tmpsection{Two or three sentences explaining what the main result reveals in direct comparison to what was thought to be the case previously}
% or how the main result adds to previous knowledge

The results conflict with each other and with other studies which suggests that tests of population structure is senstive to the exact measurements and data used.
Given the conflicting results in the literature and unclear results here, it seems likely that population structure does not strongly affect pathogen richness in bats.

\tmpsection{One or two sentences to put the results into a more general context.}

Given the weakness of any association between population structure and pathogen richness in bats, this is not a useful metric for prioritising zoonotic disease surveillance.


\tmpsection{Two or three sentences to provide a broader perspective, }
% readily comprehensible to a scientist in any discipline.



\end{abstract}

\begin{acknowledgements}
Acknowledge all the things!
\end{acknowledgements}

\setcounter{tocdepth}{2} 
% Setting this higher means you get contents entries for
%  more minor section headers.

\tableofcontents
\listoffigures
\listoftables

