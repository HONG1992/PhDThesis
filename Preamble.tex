\maketitle
\makedeclaration

\begin{abstract} % 300 word limit

\tmpsection{One or two sentences providing a basic introduction to the field}
% comprehensible to a scientist in any discipline.

\lettr{P}athogens make up a huge proportion of global diversity and their role in disease strongly affects human disease, economics and development as well as having an important ecological roles.
However, the factors that control number of pathogen species are poorly understood.


\tmpsection{Two to three sentences of more detailed background}
% comprehensible to scientists in related disciplines.

% Theory led.
% 

The patterns of contacts between individuals --- in both human and animal populations --- are nonrandom and depend on the density of individuals.
Population structure and density have important epidemiological consequences, but their role in the control of pathogen richness is unknown.




\tmpsection{One sentence clearly stating the general problem (the gap)}
% being addressed by this particular study.





\tmpsection{One sentence summarising the main result}
%  (with the words “here we show” or their equivalent).

In this thesis I have studied how population structure and density control pathogen richness using bats as a case study.
I have used epidemiological simulations and comparative analyses of data from wild bat populations to show that population structure does not have a strong affect on pathogen richness.
Using further simulations I have shown that the intereaction between population density and population structure is an important consideration.
Finally, I have created a model for estimating bat densities --- previously an incredibly challenging task --- using acoustic data.
Together, these studies clarify the relative roles of population density and structure and facilitate further study of population density in bats as a driver of pathogen richness.

\tmpsection{Two or three sentences explaining what the main result reveals in direct comparison to what was thought to be the case previously}
% or how the main result adds to previous knowledge

While theory previously predicted that population structure should increase pathogen richness, the expectation in the ecological literature was that population structure would decrease pathogen richness.
My studies support neither of these views, instead suggesting that population structure does not have a strong affect in either direction.

\tmpsection{One or two sentences to put the results into a more general context.}




\tmpsection{Two or three sentences to provide a broader perspective, }
% readily comprehensible to a scientist in any discipline.



\end{abstract}

\begin{acknowledgements}
Acknowledge all the things!
\end{acknowledgements}

\setcounter{tocdepth}{2} 
% Setting this higher means you get contents entries for
%  more minor section headers.

\tableofcontents
\listoffigures
\listoftables

