\maketitle
%\makedeclaration

\begin{abstract} % 300 word limit

\lettr{T}he huge diversity of pathogens strongly affects human health and is ecologically important.
In this thesis I examine the role of population structure and size in maintaining these high levels of diversity.
I focus on bats as a case study as they have highly varied social structures and are important reservoires for zoonotic viruses such as Ebola, SARS, Hendra and Nipah.

Firstly I test whether population structure is associated with high viral richness across wild bat species.
I find evidence that more structured bat populations have more virus species.
As this type of study cannot distinguish between specific mechanisms by which population structure might affect pathogen richness, I formulate epidemiological models to test whether structured populations may allow invading pathogens to avoid competition.
These models show that population structure does not affect the rate of pathogen invasion by this mechanism. 
Rather, in these models only the disease dynamics within the local group matter.
Given that population size is important for disease dynamics I use models to test which component of population size most strongly affects pathogen richness; group size or the number of groups.
I find that population size is more important than population density and that the important component of population abundance is group size rather than number of groups.
However, there are very few estimates of abundance for bats.
Therefore I develop a method for estimating bat abundances from acoustic surveys and use individual based simulations to validate its accurancy and precision.


Overall I show that the structure and size of populations can affect their ability to maintain a large number of pathogen species and I provide a new method to measure population sizes of bats.
These findings increases our understanding of the fundemental ecological process of pathogen community construction and can help optimize surveillance for new zoonotic pathogens.


\end{abstract}





%%%%%%%%%%%%%%%%%%%%%%%%%%%%%%%%%%%%%%%%%%%%%%%%%%%%%%%%%%%%%%
%% Acknowledgements                                         %%
%%%%%%%%%%%%%%%%%%%%%%%%%%%%%%%%%%%%%%%%%%%%%%%%%%%%%%%%%%%%%%


\begin{acknowledgements}
Acknowledge all the things!
\end{acknowledgements}

\setcounter{tocdepth}{2} 
% Setting this higher means you get contents entries for
%  more minor section headers.

\tableofcontents
\listoffigures
\listoftables

