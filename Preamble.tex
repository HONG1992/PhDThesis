\maketitle
%\makedeclaration

\begin{abstract} % 300 word limit

\lettr{T}he huge diversity of pathogens strongly affects human health and ecological systems.
I examine the role of population structure and size in maintaining this diversity.
Bats are important reservoires for zoonotic viruses such as Ebola, SARS, Hendra and Nipah and so are used as a case study throughout.

Firstly I test whether population structure is associated with high viral richness across wild bat species.
I find evidence that more structured bat populations have more virus species.
As this type of study cannot distinguish between specific mechanisms, I formulate epidemiological models to test whether structured populations may allow invading pathogens to avoid competition.
These models show that population structure does not affect the rate of pathogen invasion by this mechanism. 
Rather, in these models only the disease dynamics within the local group matter.
As both global population structure and local group size appear to be important for disease invasion, I use the same modelling framework to compare the importance of group size and number of groups.
I find that group size  has a stronger affect than number of groups.
There are very few estimates of abundance for bats to directly test the importance of population size on pathogen richness.
Therefore I develop a method for estimating bat abundances from acoustic surveys.

Overall I show that the structure and size of populations can affect their ability to maintain many pathogen species and provide a method to measure population sizes of bats.
These findings increase our understanding of the ecological process of pathogen community construction and can help optimise surveillance for zoonotic pathogens.


\end{abstract}





%%%%%%%%%%%%%%%%%%%%%%%%%%%%%%%%%%%%%%%%%%%%%%%%%%%%%%%%%%%%%%
%% Acknowledgements                                         %%
%%%%%%%%%%%%%%%%%%%%%%%%%%%%%%%%%%%%%%%%%%%%%%%%%%%%%%%%%%%%%%


\begin{acknowledgements}
Acknowledge all the things!
\end{acknowledgements}

\setcounter{tocdepth}{2} 
% Setting this higher means you get contents entries for
%  more minor section headers.

\tableofcontents
\listoffigures
\listoftables

