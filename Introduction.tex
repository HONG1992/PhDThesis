

\lettr{S}ome stuff about things. Some more things.  \blindtext

\section{General intro}

\tmpsection{There is great diversity, much of it unknown and mechanisms still not understood}
Total mammalian viral diversity.
Known mammal viruses?

Bit about how to search for viruses? Deep pcr etc.


\tmpsection{This diversity poses a zoonotic risk}

Number of known zoonotic viruses/pathogens.
Number of known from mammals.
Examples of mortality rates and examples of scale of outbreaks.

\tmpsection{Bats are a particular culprit of this risk}

Some outbreaks caused by bats.
Number of known bat viruses, known bat zoonoses.

Bats are second largest order of mammals.
Long lived, social and fly.


\section{Specific  intro}
\subsection{Pathogen richness}

Define.
Simple count of pathogens across whole species.
Ignores heterogeneity in prevalence, across space and time.

\tmpsection{Pathogen diversity is poorly understood}


\tmpsection{Known factors that affect richness}



\tmpsection{Describe structure and density}

The simple definition of animal density (animals per km$^2$) is often not a useful measure in animal species such as bats where social living is common and therefore animal density is incredibly heterogeneous.
In a bat colony animal density will be very high, areas one hundred metres outside the colony may often have a density of zero.
However we can think of density in different ways that start to capture this variety.
Firstly, simple density, as above, can be a useful metric over spatial scales broad enough to include many colonies.
Then we can also think of measuring the density of colonies (colonies per km$^2$) and average colony size.
These two factors combine to govern the broader scale animal density.

Population structure describes the fact that interactions between individuals in a population are non-rondom.
This occurs based on many factors.
Social structure, typically based on mating relationships commonly creates social structure.
Space is also likely to create structure; animals that live near each other are more likely to interact than those that live distantly.

Broader scale social structure is also very apparent in many bat species.
Bat's live in colonies, groups of animals that tend to live together in physical roosts.
Even though many species exhibit common roost switching, these colonies will often stay together.
The stability of colonies however is very variable with colonies in some species remaining constant for many years and in other species being part of a much more loose fusion-fission social system.

Bat's have a number of less common social groupings --- hibernation roosts, maternity roosts and swarming.
Also migration has important and often unknown afffects on population structure.
Do colonies survive migration?


\tmpsection{Theoretical evidence that structure and density increase richness}


\tmpsection{Empirical evidence from mammals}


\tmpsection{Pathogen diversity has an important role in zoonotic diseases and we can optimise surveillance if we understand it}

Surveillance of zoonotic diseases is crucial.
We want to categorise diseases before they spillover into humans. 
SARS was not identified until months into the pandemic for example.
We also want to anticipate outbreaks.
If we know there is an increased number of infections in host species, increased numbers of a species that is a known reservoir of a high risk zoonotic disease, or increased contacts between humans and a pathogen reservoir, we can prepare for a potential outbreak in that area.
Finally, once a pathogen does spillover, we want to respond quickly.
This is aided by having characterised the virus before it spillover, and by knowing what zoonotic pathogens we might expect in a given area.

However, funds for zoonotic surveillance are limited and so efforts must be optimised.
Knowing which bat species are likely to have many pathogens allows us to sample and identify potentially zoonotic viruses efficiently.
Furthermore, understanding the dynamics of pathogens in wild populations will allow us to know where we might expect higher chances of spillover from wild populations to humans.





