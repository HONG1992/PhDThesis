
\section{Pathogen richness and the impacts of zoonotic diseases}

\tmpsection{General intro}


\tmpsection{This diversity poses a zoonotic risk} %significance

60\% of newly emerged diseases are zoonotic (acquired from animals) with wild animals being the predominant source \cite{jones2008global, woolhouse2006host, taylor2001risk}.
Zoonotic diseases can be extremely virulent with viruses such as Nipah  and Ebola having case fatality rates over 50\% \cite{luby2009recurrent, lefebvre2014case}.
Furthermore these pathogens can have large economic costs (e.g., SARS is estimated to have cost \$40 billion \cite{knobler2004learning}).
In particular these impacts can have huge effects on developing economies.
For example, the 2014 Ebola epidemic caused both Liberia and Guinea to fall from positive to negative per capita growth rates \cite{ebolaWorldbank, ebola2015worldbank} while death rates per 1,000 people living with AIDS are up to ten times higher in developing countries than in Europe and North America \cite{granich2015trends}.

\tmpsection{Pathogen diversity has an important role in zoonotic diseases and we can optimise surveillance if we understand it}

Surveillance of zoonotic diseases is crucial to the health impacts of these diseases.
In particular it is important to categorise and describe diseases before they spill over into humans (SARS was not identified until months into the pandemic for example, \textcite{drosten2003identification}) 
It is also important to improve our ability to predict when outbreaks will occur. 
For example, if it is known that there is \textit{i})~a disease prevalence in a given host species than normal, or \textit{ii})~a greater-than-usual abundance of a species that is a known reservoir of a high risk zoonotic disease, or \textit{iii})~increased contacts between humans and a pathogen reservoir, preparations can be made for a potential outbreak in that area.

However, funds for zoonotic disease surveillance are limited and so efforts must be optimised.
Knowing which species are likely to have many pathogens allows us to sample and identify potentially zoonotic viruses efficiently.
Suggested factors that might control pathogen richness include individual, environmental and population level traits.
Individual traits that have been studied include body mass and longevity.
Increased body mass is expected to increase pathogen diversity as large bodies provide more resource for pathogens to consume and potentially more niches for them to occupy \cite{kamiya2014determines, arneberg2002host, poulin1995phylogeny} .
Increased longevity is also expected to increase pathogen richness by increasing the number of pathogens a host encounters in its lifetime \cite{nunn2003comparative, ezenwa2006host}. 
Environmental factors may also play a role. 
Latitude has been studied as a proxy for environmental factors \cite{poulin2010latitudinal, kamiya2014determines}.
It is predicted that warmer climates promote species richness via metabolic mechanisms or by increasing the rate of evolution \cite{brown2004toward, dunn2010global, rohde1992latitudinal}.
Furthermore, population level traits that affect the dynamics of disease spread have also been studied.
Animal density \cite{kamiya2014determines, nunn2003comparative, arneberg2002host}, sociality \cite{bordes2007rodent, vitone2004body, altizer2003social, ezenwa2006host} and population structure \cite{nunes2006localized, maganga2014bat, gay2014parasite, turmelle2009correlates} have both been predicted to increase pathogen richness by increasing the rate of spread of new pathogens.
Finally, species with larger range sizes are expected to have higher pathogen richness as they experience a wider range of environments and have more sympatric host species \cite{kamiya2014determines, nunn2003comparative}).
These relationships provide a basis for predicting which species will have high pathogen richness and should be prioritised for sampling and surveillance.
However, without a better mechanistic understanding of how pathogen richness is created and maintained it is difficult to predict how pathogen richness, and therefore zoonotic disease risk, will respond to global change.

\tmpsection{There is great diversity, much of it unknown and mechanisms still not understood} % context

The global richness of pathogens is large but mostly unknown \cite{poulin2014parasite}.
Recent large studies have found tens \cite{anthony2013strategy} or even hundreds \cite{anthony2015non} of virus species in a single host species.
This suggests that the global number of mammalian virus species is of the order of hundreds of thousands \cite{anthony2013strategy} while only 3,000 virus species, across all host groups, are currently described \cite{ICTV}.
Recent large databases include nearly 2,000 pathogens from approximately 400 wild animal hosts \cite{wardeh2015database}.
Given that there are over 5,000 named mammal species \cite{wilson2005mammal}, the undiscovered diversity of pathogens is likely huge.

Competition between pathogens can occur by different mechanisms: immunological mechanisms such as cross-immunity or shared immune response \cite{fenton2010applying} and ecological mechanisms such as removal of susceptible hosts by death \cite{rohani2003ecological} or competition for internal host resources \cite{griffiths2014analysis}.
Like ecological systems this competition leads us to expect competitive exclusions so the diversity of parasites needs an explanation \cite{bremermann1989competitive, martcheva2013competitive, ackleh2003competitive, ackleh2014robust, turner2002impact}.



\section{Influence of population structure and size on pathogen richness}


\tmpsection{Theoretical evidence that structure and density increase richness}

% Density

The role of population size and density in disease dynamics is well established \cite{may1979population, anderson1979population, heesterbeek2002brief, lloyd2005should}.
Broadly, larger populations can maintain diseases more easily, due to having a larger pool of susceptible individuals (individuals without acquired immunity) and having a greater number of new susceptible individuals enter the population by birth or immigration \cite{may1979population, anderson1979population}.
High density populations are expected to have a greater number of contacts between individuals and so promote disease spread (though there is much discussion on when the number of contacts might scale independently of density \cite{mccallum2001should}).

% Structure

There is a large literature on the role of population structure on disease dynamics (see review by \textcite{pastor2015epidemic}) driven by applications to computer viruses \cite{pastor2001epidemic} and the social spread of information \cite{goffman1964generalization}, as well as health applications.
In particular, work has concentrated on how structure affects the invasion threshold, $R_0$ \cite{colizza2007invasion, barthelemy2010fluctuation, wu2013threshold, may2001infection, pastor2001epidemic}. 
This value combines relevant parameters to yield a threshold above which a disease is expected to infect a significant proportion of the population \cite{may1979population, anderson1979population}.
Below the threshold, only small outbreaks that quickly die out are expected.


However, the majority of theoretical work considers single pathogens with models examining whether a pathogen can spread and persist in a population, ignoring all other pathogens.
A number of studies examine competing epidemics --- when two pathogens spread at the same time, which pathogen infects more individuals? --- and have found increasing population structure reduces dominance of one strain \cite{van2014domination, poletto2013host, poletto2015characterising}.
However, this again tells us little about how pathogen communities form and what factors control total pathogen richness.
Far fewer papers explicitly study the longer term competition between two or more pathogens.
Those that do commonly find that competitive exclusion is likely \cite{castillo1995dynamics, bremermann1989competitive, martcheva2013competitive, ackleh2003competitive, ackleh2014robust, turner2002impact}.
Mechanisms that have been shown to allow pathogen coexistence include superinfection \cite{may1994superinfection, li2010age}, density-dependent deaths \cite{ackleh2003competitive, kirupaharan2004coexistence} and differing transmission routes \cite{allen2003dynamics}.

The specific role of density on the ability of pathogens to coexist has not been theoretically studied though it is commonly found to promote pathogen richness in comparative empirical studies \cite{kamiya2014determines, nunn2003comparative, arneberg2002host}.
The few papers that have directly studied how coexistence of pathogens responds to population structure have found that population structure can allow pathogens to coexist when competitive exclusion would occur in a fully mixed population \cite{qiu2013vector, allen2004sis, nunes2006localized}.
Furthermore, genetic diversity has been shown to be maximised at intermediate levels of population structure \cite{campos2006pathogen}.
The roles of population structure and social group size have been examined in comparative studies \cite{maganga2014bat, gay2014parasite, turmelle2009correlates, altizer2003social, bordes2007rodent, ezenwa2006host, rifkin2012animals, vitone2004body}.
There is much disagreement between these studies with population structure being shown to promote \cite{maganga2014bat, turmelle2009correlates} and inhibit pathogen richness \cite{gay2014parasite} and similarly group size being shown to promote \cite{rifkin2012animals, bordes2007rodent} and inhibit \cite{ezenwa2006host} richness.
While increased group size should generally decrease population structure the literature is rarely clear on how these variables relate.



\tmpsection{Specific intro}

\section{Bats as reservoirs of zoonotic diseases}

\tmpsection{Bats are a particular culprit of this risk, and have unknown diversity and interesting population structure.} %context

In recent decades bats have been implicated in a number of high profile zoonotic outbreaks including Nipah \cite{field2001natural, halpin2011pteropid}, Ebola \cite{leroy2005fruit}, SARS \cite{li2005bats} and Hendra \cite{field2001natural}.
This has lead to much research on whether bats are a particular source of zoonotic disease \cite{luis2013comparison, olival2015bats, wang2011mass} and examinations of factors, such as flight, social living and longevity, that might predispose them to being reservoirs of zoonotic viruses \cite{calisher2006bats, o2014bat, dobson2005links, racey2015uniqueness, kuzmin2011bats}.
Given that bats are the second largest order of mammals \cite{wilson2005mammal}, we may expect them to be the source of many viruses simply through weight of numbers \cite{luis2013comparison}.
The broad conclusions are that while bat do host more zoonotic viruses than other groups \cite{luis2013comparison} they do not host more virus species per host species \cite{olival2015bats}.
%Questions remain as to why bat viruses have a tendency to have such high zoonotic potential.

Many factors of bat populations make them epidemiologically interesting.
They have highly varied and sometimes complex social structures \cite{kerth2008causes}.
While some species are largely solitary or live in very small groups (e.g., \emph{Lasiurus borealis} \cite{shump1982lasiurus}) some species live in colonies of millions of individuals (e.g., \emph{Pteropus scapulatus} \cite{birt2008little}).
These groups can be very stable \cite{kerth2011bats, mccracken1981social}.
Further complexity arises due to their propensity for seasonal migration \cite{fleming2003ecology, richter2008first, cryan2014continental} and seasonally changing social organisation including maternity roosts, hibernation roosts and swarming sites \cite{kerth2008causes}.
Finally, their ability to fly means populations can be well mixed across large distances \cite{peel2013continent, petit1999male} but this is highly variable with some species having limited dispersal \cite{wilmer1994extreme}.

However, the population density of many bat species, particularly tree roosting species, is unknown \cite{clement2013estimating}.
As they are small, nocturnal and difficult to identify in flight, estimating their density is incredibly difficult without disruptive and time-consuming roost surveys \cite{kloepper2016estimating, humphrey1971photographic, sabol1995technique}.
Furthermore, bat densities are generally estimated by counting bats in roosts and dividing this number by area which assumes all roosts have been surveyed \cite{speakman1991minimum, zahn2006population, moreno2004colony}. 
As density is tightly linked to pathogen richness \cite{kamiya2014determines} and central to epidemiological models \cite{may1979population, anderson1979population} this leaves large gaps in our understanding of disease processes in this group.

\tmpsection{Bats are hard to detect etc.}




\section{Thesis overview}

In this thesis I examined the role of population structure and density on pathogen richness.
I used bats as a case study throughout due to their interesting social structure and importance as zoonotic	reservoirs.
I combined empirical, comparative studies with simulation models.
This allowed me to study specific mechanisms while linking my theoretical insights to real-world, empirical tests of hypotheses.


\tmpsection{Chapter 2}

First, in Chapter \ref{ch:empirical}, I empirically test the hypothesis that population structure is associated with pathogen richness in wild bat populations.
I used two measures of population structure --- the number of subspecies and gene flow --- and a larger data set than previous studies to ensure robust results and used viral richness as a proxy for overall pathogen richness.
For both measures I found that bat species with more structured populations have more known viruses.
This relationship is robust to controlling for study bias and phylogenetic nonindependence.
I also tested for relationships between body mass and pathogen richness, and range size and pathogen richness, finding strong support for larger bodied bats carrying more viruses and mixed support for range size promoting pathogen richness.

\tmpsection{Chapter 3}

In Chapter \ref{ch:sims1}, I examined one specific mechanism by which population structure may promote increased pathogen richness.
I tested whether increased population structure can allow newly evolved pathogen strains to invade and persist more easily.
I modelled bat populations as individual-based, stochastic meta-populations and examined the competition dynamics of two identical pathogen strains.
I tested two factors related to population structure: dispersal rate and the number of links between subpopulations.
I found that neither of these factors significantly affected the probability of newly evolved pathogens invading and persisting in the population.

\tmpsection{Chapter 4}

Next, I examined the relationships between a number of elements of population structure (Chapter \ref{ch:sims2}).
I clarified the interdependence between range size, population size and density.
I also noted that population size can be decomposed into colony size and the number of colonies.
Using the same model as in Chapter \ref{ch:sims1}, I then tested which of these factors are most important in promoting pathogen richness.
Specifically I tested which factor most strongly promotes the invasion and establishment of newly evolved pathogens.
I found that population size is more important than population density and that colony size is the important component of population size.

\tmpsection{Chapter 5}

Given the importance of population size on pathogen richness it is important to have good population estimates for wild bat populations.
However, there are currently very few measurements of bat population size due to their small size, nocturnal habit and difficulties in identification.
Therefore I aimed to develop a method for estimating bat population size from acoustic data, specifically data collected by the iBats project \cite{jones2011indicator}.
In Chapter \ref{ch:grem} I present a generally applicable method --- based on random encounter models \cite{rowcliffe2008estimating, yapp1956theory} --- for estimating population sizes of animal populations using camera traps or acoustic detectors.
I used spatial simulations to test the method for biases and to assess its precision.
I found that the method is unbiased and precise as long as a reasonable amount of data is collected.


\tmpsection{Chapter 6: Conclusions}

%to do Conclusions chapter
Finally, in Chapter \ref{ch:discussion}, I discuss broader conclusions, applications and implications of my results.
I also discuss potential future directions for research.












