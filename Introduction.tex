

\tmpsection{General intro}

\tmpsection{There is great diversity, much of it unknown and mechanisms still not understood} % context

\lettr{T}he diversity of pathogens is huge and largely unknown \cite{poulin2014parasite}.
Recent large studies have found tens \cite{anthony2013strategy} or even hundreds of virus species in a single host species \cite{anthony2015non}.
This suggests the number of mammalian viruses globally is of the order of hundreds of thousands of virus species \cite{anthony2013strategy} while only three thousand species of virus, across all host groups, are currently described \cite{ICTV}.
Recent large databases include nearly 2,000 pathogens from approximately 400 wild animal hosts \cite{wardeh2015database}.
Given that there are nearly 4,000 named mammal species \cite{wilson2005mammal}, the undiscovered diversity of pathogens is likely huge.

\tmpsection{This diversity poses a zoonotic risk} %significance

This diversity of pathogens presents a risk to human health.
60\% of newly emerged disease are zoonotic (acquired from animals) with wild animals being the predominant source \cite{jones2008global, woolhouse2006host, taylor2001risk}.
Zoonotic diseases can be extremely virulent: viruses such as Nipah \cite{luby2009recurrent}, Ebola \cite{lefebvre2014case} having case fatality rates over 50\%.
Furthermore these pathogens can have large economic costs (e.g. SARS is estimated to have cost \$40 billion \cite{knobler2004learning}).
In particular these impacts can have huge affects on developing economies.
For example the 2014 ebola epidemic caused both Liberia and Guinea to fall from positive to negative per capita growth rates \cite{ebolaWorldbank, ebola2015worldbank}.
In general we can expect competition between pathogens.
Competition between pathogens can occur by different mechanisms: immunological mechanisms such as cross-immunity or shared immune respone \cite{fenton2010applying} and ecological mechanisms such as removal of susceptable hosts by death \cite{rohani2003ecological} or competition for internal host resources \cite{griffiths2014analysis}.
Like ecological systems this competition leads us to expect competitive exclusions \cite{bremermann1989competitive, martcheva2013competitive, ackleh2003competitive, ackleh2014robust, turner2002impact} so the diversity of parasites needs an explanation.

\tmpsection{Pathogen diversity has an important role in zoonotic diseases and we can optimise surveillance if we understand it}

Surveillance of zoonotic diseases is crucial.
In particular we want to categorise diseases before they spillover into humans (SARS was not identified until months into the pandemic for example \cite{drosten2003identification}) and we want to anticipate outbreaks. 
If we know there is an increased number of infections in host species, increased numbers of a species that is a known reservoir of a high risk zoonotic disease, or increased contacts between humans and a pathogen reservoir, we can prepare for a potential outbreak in that area.
However, funds for zoonotic surveillance are limited and so efforts must be optimised.
Knowing which species are likely to have many pathogens allows us to sample and identify potentially zoonotic viruses efficiently.
Suggested factors that might control pathogen richness include individual traits (body mass \cite{kamiya2014determines, arneberg2002host, poulin1995phylogeny} and longevity \cite{nunn2003comparative, ezenwa2006host})
as well as environmental factors such as latitude \cite{poulin2010latitudinal, kamiya2014determines}.
Further population level traits that affect the dynamics of disease spread have also been studied (animal density \cite{kamiya2014determines, nunn2003comparative, arneberg2002host}, sociality \cite{bordes2007rodent, vitone2004body, altizer2003social, ezenwa2006host}, population structure \cite{nunes2006localized, maganga2014bat, gay2014parasite, turmelle2009correlates} and  species range size \cite{kamiya2014determines, nunn2003comparative}).
This understanding provides a basis for predicting which species will have high pathogen richness and should be prioritised for sampling and surveillance.
Furthermore, given a good mechanistic understanding of how pathogen richness is created and maintained we can start to predict how pathogen richness (and zoonotic disease risk) will respond to global change.



\tmpsection{Specific intro}


\tmpsection{Bats are a particular culprit of this risk, and have unknown diversity and interesting population structure.} %context

In recent decades bats have been implicated in a number of high profile zoonotic outbreaks including Nipah \cite{field2001natural}, Ebola \cite{leroy2005fruit}, SARS \cite{li2005bats} and Hendra \cite{field2001natural}.
This has lead to much research on whether bats are a particular source of zoonotic disease \cite{luis2013comparison, olival2015bats, wang2011mass} and examinations of factors, such as flight, social living and longevity, that might predispose them to being reservoires of zoonotic viruses \cite{calisher2006bats, o2014bat, dobson2005links, racey2015uniqueness}.
Given that bats are the second largest order of mammals \cite{wilson2005mammal}, we may expect them to be the source of many viruses simply through weight of numbers \cite{luis2013comparison}.
The broad conclusions are that while bat do host more zoonotic viruses than other groups \cite{luis2013comparison} they do not host more virus species in general \cite{olival2015bats}.
Questions remain as to why bat viruses have a tendancy to have such high zoonotic potential.

Many factors of bat populations make them epidemiologically interesting.
Firstly, the population density of most bat species is largely unknown.
As they are small, noctornal and difficult to identify on the wing, estimating there density is incredibly difficult without distruptive and time consuming roost surveys.
As this parameter is tightly linked to pathogen richness \cite{kamiya2014determines} and central to epidemiological models \cite{may1979population, anderson1979population} these leaves large gaps in our understanding of disease processes in this group.
Secondly they have highly varied and sometimes complex social structures \cite{kerth2008causes}.
While some species are largely solitary or live in very small groups (e.g. \emph{Lasiurus borealis} \cite{shump1982lasiurus}) some species live in colonies of millions of individuals (e.g. \emph{Pteropus scapulatus} \cite{birt2008little}).
These groups can be very stable \cite{kerth2011bats, mccracken1981social}.
Further complexity arises due to their propensity for seasonal migration \cite{fleming2003ecology, richter2008first, cryan2014continental} and seasonally changing social organisation including maternity roosts, hibernation roosts and swarming sites \cite{kerth2008causes}.
Finally, their ability to fly means populations can be well mixed across large distances \cite{peel2013continent, petit1999male} but this is highly variable with some species having limited dispersal \cite{wilmer1994extreme}.




%Bit about how to search for viruses? Deep pcr etc.


\tmpsection{Pathogen richness}

Define.
Simple count of pathogens across whole species.
Ignores heterogeneity in prevalence, across space and time.







\tmpsection{Theoretical evidence that structure and density increase richness}

% Density



% Structure

\tmpsection{Empirical evidence from mammals}



\tmpsection{Thesis overview}



\tmpsection{Chapter 2}


\tmpsection{Chapter 3}


\tmpsection{Chapter 4}


\tmpsection{Chapter 5}





