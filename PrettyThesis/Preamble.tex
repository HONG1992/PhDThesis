\maketitle
\makedeclaration

\begin{abstract} % 300 word limit

\lettr{P}athogens make up a huge proportion of global diversity and their role in disease strongly affects human disease, economics and development as well as having an important ecological role.
However, the factors that control the number of pathogen species are poorly understood.
In this thesis I examine the role population structure and density in maintaining these high levels of diversity in the face of interpathogen competition.
Throughout the thesis I focus on bats (Order: Chiroptera) as a case study for more general disease processes as they have highly varied social structures and have emerged as important reservoires for zoonotic viruses such as Ebola, SARS, Hendra and Nipah.

In Chapter \ref{ch:empirical} I test whether population structure is associated with high viral richness in wild bat species.
I find evidence that more structured bat populations --- measured by the number of subspecies and gene flow estimates --- typically have more virus species.
Body mass is also found to strongly affect virus richness while I find contradictory support for range size as a predictor of viral richness.
As comparative studies --- as used in Chapter \ref{ch:empirical} --- cannot distinguish between specific mechanisms by which population structure might affect pathogen richness, I next used simulated epidemiological models to test whether structured population may allow invading pathogens to avoid competition and so establish themselves in the population.
I present the results of these simulations in Chapter \ref{ch:sims1} and find that in populations parameterised to mimic bat populations population structure neither promotes not hinders invasion and establishment of pathogens.
Instead it is the factors that affect disease spread at the very earliest stages of the epidemic, namely transmission rate, that determine whether or not a new pathogen will invade.
One of the factors that strongly affects disease transmission at the very beginning of an epidemic is the size of the subpopulation.
Furthermore, one of the factors known to affect pathogen richness in many taxa is population density, which can be decomposed into group size and number of groups in an area.
To clarify the interelations between these measures, and to test which most strongly affects pathogen richness I ran simulations varying group, number of groups, population abundance and population density seperately (Chapter \ref{ch:sims2}) and comparing how they affec the probability of invasion of new pathogens in the presence of pathogen competition.
I find that population abundance is more important that population density and that the important component of population abundance is group size rather than number of groups.
The importance of population abundance is highlighted in Chapter \ref{ch:sims2} and in the literature but there are very few estimates of abundance for bats.
In Chapter \ref{ch:grem} I present a method for estimating bat abundances from acoustic survays and use individual based simulations to validate its accurancy and precision.
This method has the potential to make estimating bat abundances much easier than previously and these estimates will be vital in expanding our understanding of disease dynamics in bat populations.

Overall I show that the structure and size of populations, in particular bat populations, can affect their ability to maintain a large number of pathogen species and I provide a new method to measure population sizes of bats.
These findings increases our understanding of the fundemental ecological process of pathogen community construction and the results and new methods provided can help optimize surveillance for new zoonotic pathogens.


\end{abstract}



%%%%%%%%%%%%%%%%%%%%%%%%%%%%%%%%%%%%%%%%%%%%%%%%%%%%%%%%%%%%%%
%% Acknowledgements                                         %%
%%%%%%%%%%%%%%%%%%%%%%%%%%%%%%%%%%%%%%%%%%%%%%%%%%%%%%%%%%%%%%

\begin{acknowledgements}
Acknowledge all the things!
\end{acknowledgements}

\setcounter{tocdepth}{0} 
% Setting this higher means you get contents entries for
%  more minor section headers.

\tableofcontents
\listoffigures
\listoftables

