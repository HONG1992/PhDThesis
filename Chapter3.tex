\clearpage







\section{Abstract}


\tmpsection{One or two sentences providing a basic introduction to the field}
% comprehensible to a scientist in any discipline.
\lettr{I}t is still unclear what factors determine the number of pathogens a wild species carries.
But once understood, these factors could provide a way to prioritise surveillance of wild populations for zoonotic disease.


\tmpsection{Two to three sentences of more detailed background}
% comprehensible to scientists in related disciplines.

% Theory led.
% 

The pattern of contacts between individuals (i.e. population structure) has long been known to strongly affect epidemic processes.
Theory suggests that population structure can promote pathogen richness while the ecological literature generally assumes it will decrease richness.
It is still unclear how important population structure is in controlling pathogen richness in wild populations.
Previous studies have had contradictory results and the different measures of population structure have different shortcomings.


\tmpsection{One sentence clearly stating the general problem (the gap)}
% being addressed by this particular study.

Here I use comparative data to test whether population structure influences pathogen richness in bats.
I use two measures of population structure: a novel measure, number of subspecies, and a more careful application of genetic measures which have been used previously.

\tmpsection{One sentence summarising the main result}
%  (with the words “here we show” or their equivalent).

I find conflicting evidence suggesting that while population structure may promote pathogen diversity it is likely not a strong affect.


\tmpsection{Two or three sentences explaining what the main result reveals in direct comparison to what was thought to be the case previously}
% or how the main result adds to previous knowledge

The results conflict with each other and with other studies which suggests that tests of population structure is senstive to the exact measurements and data used.
Given the conflicting results in the literature and unclear results here, it seems likely that population structure does not strongly affect pathogen richness in bats.

\tmpsection{One or two sentences to put the results into a more general context.}

Given the weakness of any association between population structure and pathogen richness in bats, this is not a useful metric for prioritising zoonotic disease surveillance.


\tmpsection{Two or three sentences to provide a broader perspective, }
% readily comprehensible to a scientist in any discipline.




%%%%%%%%%%%%%%%%%%%%%%%%%%%%%%%%%%%%%%%%%%%%%%%%%%%%%%%%%%%%%%%%%%%%%%%%%%%%%%%%%%%%%%%%%%%%%%%%%%%%%%%%%%%%%%%%%%%%%%%%%%%%%%%%%%%%%%%%%%%%%%%%%%%%%%%%%%%

\clearpage
\section{Introduction}

%%%%%%%%%%%%%%%%%%%%%%%%%%%%%%%%%%%%%%%%%%%%%%%%%%%%%%%%%%%%%%%%%%%%%%%%%%%%%%%%%%%%%%%%%%%%%%%%%%%%%%%%%%%%%%%%%%%%%%%%%%%%%%%%%%%%%%%%%%%%%%%%%%%%%%%%%%%
\subsection{General Intro}




\subsection{Specific Intro}



\subsection{Theoretical background}

\subsection{Previous Studies}

Three studies have used comparative data to test for an association between population structure and viral richness.
A study on 15 African bats found a positive relationship between distribution fragmentation and viral richness \cite{maganga2014bat} while a study on 20 South-East Asian bats found the opposite relationship \cite{gay2014parasite}. 
A global study on 33 bats found a positive relationship between $F_{ST}$ --- a measure of genetic structure --- and viral richness \cite{turmelle2009correlates}. 
However, this study included measures using mtDNA which only measures female dispersal which may haved biased the results many bat species show female philopatry \cite{kerth2002extreme, hulva2010mechanisms}.
Furthermore, this study used measures of $F_{ST}$ irrespective of the study scale with studies covering from tens \cite{mccracken1981social} to thousands \cite{petit1999male} of kilometers.
As isolation by distance has been shown in a number of bat species \cite{burland1999population, hulva2010mechanisms, o2015genetic, vonhof2015range} this could bias results further.
Finally, when a global $F_{ST}$ value is not given they use the mean of all pairwise $F_{ST}$ between sites.
It is not clear that this is correct as from global $F_{ST}$ we expect migration rates of $M = \frac{1-F_{ST}}{8F_{ST}}$ while from $F_{ST}$ between pairs of populations we expect migration rates of $M = \frac{1-F_{ST}}{8F_{ST}}$ where $M$ is the absolute number of diploid inviduals dispersing per generation \cite{slatkin1995measure}.
As it is in fact the movement of individuals that is epidemiologically relavent, using these studies is probably not correct without attempting to correct for these difference.



\subsection{Rates}



\subsection{Choice of measure of population structure}

\tmpsection{Direct dispersal measurements}

\tmpsection{Measures from range}

\tmpsection{Genetic measures}

\tmpsection{Number of Subspecies}





\subsection{The gap}




\subsection{What I did}




\subsection{What I found}




%%%%%%%%%%%%%%%%%%%%%%%%%%%%%%%%%%%%%%%%%%%%%%%%%%%%%%%%%%%%%%%%%%%%%%%%%%%%%%%%%%%%%%%%%%%%%%%%%%%%%%%%%%%%%%%%%%%%%%%%%%%%%%%%%%%%%%%%%%%%%%%%%%%%%%%%%%%

%\clearpage
\section{Methods}

%%%%%%%%%%%%%%%%%%%%%%%%%%%%%%%%%%%%%%%%%%%%%%%%%%%%%%%%%%%%%%%%%%%%%%%%%%%%%%%%%%%%%%%%%%%%%%%%%%%%%%%%%%%%%%%%%%%%%%%%%%%%%%%%%%%%%%%%%%%%%%%%%%%%%%%%%%%




































































































































