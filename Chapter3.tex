





\section{Abstract}


%\tmpsection{One or two sentences providing a basic introduction to the field}
% comprehensible to a scientist in any discipline.
\lettr{Z}oonotic diseases make up a majority of human infectious diseases and are a major drain on healthcare resources and economies.
Species that host many pathogen species are more likely to be the source of a novel zoonotic disease than species with few pathogens.
However, the factors that influence pathogen richness in animal species are poorly understood.
%
%
%\tmpsection{Two to three sentences of more detailed background}
% comprehensible to scientists in related disciplines.
% Theory led.
The pattern of contacts between individuals (i.e. population structure) can be influenced by habitat fragmentation, sociality and dispersal behaviour.
Epidemiological theory suggests that population structure can promote pathogen richness by reducing competition between pathogen species.
Conversely, it is often assumed that as population structure slows the spread of a new pathogen, less structured populations should have greater pathogen richness.
%
%
%\tmpsection{One sentence clearly stating the general problem (the gap)}
% being addressed by this particular study.
Previous studies have had contradictory results and different measures of population structure have been used complicating the interpretation.
%
%
%\tmpsection{One sentence summarising the main result}
%  (with the words “here we show” or their equivalent).
Here I use comparative data across 203 bat species using phylogenetic linear models, controlling for body mass, range size and study effort, to test whether population structure correlates with pathogen richness.
I use bats as a case study as they have been associated with a number of important, recent zoonotic outbreaks.
Unlike previous studies I used two measures of population structure: the number of subspecies, and a more careful application of genetic measures than have been used previously.
Both measures are positively associated with pathogen richness.
%
%
%\tmpsection{Two or three sentences explaining what the main result reveals in direct comparison to what was thoughts to be the case previously}
% or how the main result adds to previous knowledge
My results add more robust support to the hypothesis that population structure promotes pathogen richness in bats.
The results support the prediction that population structure reduces competition between pathogens and so allow greater pathogen richness and
contradict the prediction that factors that increase $R_0$ should increase pathogen richness.
This implies that competitive processes amongst pathogens are stronger than previously thought.
%
%
%\tmpsection{One or two sentences to put the results into a more general context.}
Although my analysis implies that population structure does promote pathogen richness in bats, the weakness of the relationship and the difficulty in obtaining some measurements means this is probably not a useful, predictive factor on its own for optimising zoonotic surveillance.
However, the relationship has implications for global change, implying that increased habitat fragmentation might promote greater viral richness in bats.





%%%%%%%%%%%%%%%%%%%%%%%%%%%%%%%%%%%%%%%%%%%%%%%%%%%%%%%%%%%%%%%%%%%%%%%%%%%%%%%%%%%%%%%%%%%%%%%%%%%%%%%%%%%%%%%%%%%%%%%%%%%%%%%%%%%%%%%%%%%%%%%%%%%%%%%%%%%

\section{Introduction}

%%%%%%%%%%%%%%%%%%%%%%%%%%%%%%%%%%%%%%%%%%%%%%%%%%%%%%%%%%%%%%%%%%%%%%%%%%%%%%%%%%%%%%%%%%%%%%%%%%%%%%%%%%%%%%%%%%%%%%%%%%%%%%%%%%%%%%%%%%%%%%%%%%%%%%%%%%%

%#the introduction is not bad and starts very well but i think you need a bit more from studies of other mammals (not bats) to put the study into context as well as explaining why particularly you focus on pop structure, some justification of why bats, and less detail about the specific Fst measures (move to methods) and more stuff on your actual methods and approach you use in this study.

%#Structure could be:
%#1. Zoonotic disease is bad (as you have written it already)
%#2. Need to understand why some species have more pathogens than others. Life history variables of the host have been used to explain why some species have more than others, such as blah blah. However, pop structure (explain what this means) is of particular interest because of blah blah.
%#3. Epidemiological theoretical models predict relationship with pop structure and translated into across species patterns as increased structure less pathogen diversity but problem is of inter-pathogen competition
%#4. lack of large across species studies of these relationships - those that have been done have conflicting patterns (examples across different taxa).
%#5. Bats are very interesting in this regard because of blah
%#6. Bat studies of pathogen richness and population structure are particularly interesting in this area but also are conflicting (examples), due in part to low sample sizes and problems with comparing results using different definitions of population structure and not controlling for effects of phylogeny.
%#7. Here I use a phylogenetic comparative approach to understand the relationship between pop structure and pathogen richness across the largest study of bats to date. I use a phylogenetic GLM controlling for the other life history characteristics known to impact pathogen richness to quantify the relationship between viral richness (as a proxy for pathogen richness_ and two measures of population structure. 
%#8. I found ...

\tmpsection{General Intro}

%#1. Zoonotic disease is bad (as you have written it already)
Zoonotic pathogens make up the majority of newly emerging diseases and have profound consequences for public health, economics and international development \cite{jones2008global, smith2014global, ebolaWorldbank}.
Better predictive models of which wild host species are potential reservoirs of zoonotic diseases would allow us to effectively optimise zoonotic disease surveillance and anticipate how the risks of disease spillover might change with global change.
The chance that a host species will be the source of an outbreak depends on a number of factors including its proximity and interactions with humans and the prevalence and the diversity of pathogens it carries \cite{wolfe2000deforestation}.
However, our understanding of the factors that control the number of pathogen species in a host species is still poor.


\tmpsection{Specific Intro}

%#2. Need to understand why some species have more pathogens than others. Life history variables of the host have been used to explain why some species have more than others, such as blah blah. 
\tmpsection{Theoretical background}


A number of individual and population level traits that might control pathogen richness have been studied.
Individual traits that have been studied include body mass and longevity.
Large bodied animals have been shown to have high pathogen richness with large bodies providing more resource for pathogens \cite{kamiya2014determines, arneberg2002host, poulin1995phylogeny}.
Long lived species are expected to have high pathogen richness as the number of pathogens a host encounters in its lifetime will be higher \cite{nunn2003comparative, ezenwa2006host}. 
Population level traits have also been studied.
Animal density \cite{kamiya2014determines, nunn2003comparative, arneberg2002host} and sociality \cite{bordes2007rodent, vitone2004body, altizer2003social, ezenwa2006host} are both predicted to increase pathogen richness by increasing the rate of spread, $R_0$, of a new pathogen.
Finally, widely distributed species have high pathogen richness, potentially because they experience a wider range of environments or because they are sympatric with host species \cite{kamiya2014determines, nunn2003comparative}).

%# However, pop structure (explain what this means) is of particular interest because of blah blah.

%#3. Epidemiological theoretical models predict relationship with pop structure and translated into across species patterns as increased structure less pathogen diversity but problem is of inter-pathogen competition


A further population level factor that may affect pathogen richness is population structure.
Population structure can be defined as the extent to which interactions between individuals in a population are non-random.
The role of population structure on human epidemics has been studied in depth and it has been shown that decreased population structure increases the speed of disease spread and makes establishment of a new pathogen more likely \cite{colizza2007invasion, vespignani2008reaction}.
In comparative studies of pathogen richness in wild animals, this relationship with $R_0$ is often taken as a prediction that decreased population structure will increase pathogen richness relative to other host species \cite{nunn2003comparative, morand2000wormy, poulin2014parasite, poulin2000diversity, altizer2003social}. 
However, epidemiological models of highly virulent pathogens have shown that increased population structure can allow persistence of a pathogen where a well-mixed population would experience a single, large epidemic followed by pathogen extinction \cite{blackwood2013resolving, plowright2011urban}.
Furthermore, the assumption that high $R_0$ leads to high pathogen richness ignores inter-pathogen competition.
Simple epidemiological models of competition between multiple pathogens show that in unstructured populations a competitive exclusion process occurs but that adding population structure allows coexistence \cite{qiu2013vector, allen2004sis, nunes2006localized}.





\tmpsection{Previous Studies}

%#4. lack of large across species studies of these relationships - those that have been done have conflicting patterns (examples across different taxa).

%#5. Bats are very interesting in this regard because of blah

%#6. Bat studies of pathogen richness and population structure are particularly interesting in this area but also are conflicting (examples), due in part to low sample sizes and problems with comparing results using different definitions of population structure and not controlling for effects of phylogeny.

Three studies have used comparative data to test for an association between population structure and viral richness.
A study on 15 African bat species found a positive relationship between the extent of distribution fragmentation and viral richness \cite{maganga2014bat}.
Conversely, a study on 20 South-East Asian bat species found the opposite relationship \cite{gay2014parasite}. 
These studies used the ratio between the perimeter and area of the species' geographic range as their measure of population structure.
Range maps are very course for many species.
Furthermore there is a potential bias with island living species being given sea based edges where continental species might be assumed to live across their entire range, without considering the different terrestrial habitats in these areas.

A global study on 33 bat species found a positive relationship between $F_{ST}$ --- a measure of genetic structure --- and viral richness \cite{turmelle2009correlates}. 
However, this study included measures using mtDNA which only measures female dispersal which may have biased the results as many bat species show female philopatry \cite{kerth2002extreme, hulva2010mechanisms}.
Furthermore, this study used measures of $F_{ST}$ irrespective of the study scale including studies covering from tens \cite{mccracken1981social} to thousands \cite{petit1999male} of kilometres.
As isolation by distance has been shown in a number of bat species \cite{burland1999population, hulva2010mechanisms, o2015genetic, vonhof2015range} this could bias results further.
Finally, when a global $F_{ST}$ value is not given they used the mean of all pairwise $F_{ST}$ values between sites.
This is not correct as pairwise and global $F_{ST}$ values have different relationships with effective migration rates. 

%Global $F_{ST}$ should be calculated as
%global $F_{ST}$ values we expect migration rates of $M = (1-F_{ST})/4F_{ST}$ while from $F_{ST}$ values between pairs of populations we expect migration rates of $M = (1-F_{ST})/8F_{ST}$ where $M$ is the effective number of diploid individuals dispersing per generation \cite{slatkin1995measure}.
%To use studies that only present pairwise $F_{ST}$ values the raw data would have to be gathered and global $F_{ST}$ calculated from those.
%As it is in fact the movement of individuals that is epidemiologically relevant, using these studies is probably not accurate without attempting to correct for these differences.



%\tmpsection{Choice of measure of population structure}
%
%A number of measurements of population structure have been used in the literature and each has its own shortcomings.
%In particular, the better, more direct measurements tend to be very work intensive which consequently means data is available for few species.
%
%\tmpsection{Direct dispersal measurements}
%
%The ideal metric of population structure is direct measurement of dispersal rates and distance.
%These are incredibly difficult to obtain, especially over large scales.
%Some very large mark-recapture studies have been conducted, but recapture rates are low \cite{norquay2013long}.
%In practise, direct measurements are not practical for comparative analysis due to the lack of data and inconsistency in data collection methods.
%%todo
%
%\tmpsection{Genetic measures}
%
%As direct measurement of dispersal is difficult, genetic data is often used.
%Measurement such as $F_{ST}$ are used to calculate migration.
%There are strong model assumptions under the conversion from $F_{ST}$ to migration.
%Furthermore, estimates are affected by the spatial spatial scale of the studies and the genetic regions being sequenced.
%These differences should not be ignored.
%However, the main issue with this measure is the effort required for each study and the subsequent lack of data.
%
%\tmpsection{Number of Subspecies}
%
%For a population to evolve distinct phenotypic or genetic traits, such that they can be classed as a subspecies, there must be limited migration between populations.
%The number of subspecies a species has therefore reflects the level of population structure in that species.
%The value of this measurement is available for every bat species.
%However, it is likely biased, with well studied species being likely to have more recognised subspecies.
%Further, this is a very course measure and it is important to consider whether it is measuring migration at a timescale and rate that is epidemiologically relevant.
%
%For both measures from $F_{ST}$ and the number of subspecies it is useful to consider the rates of animal movement that are being measured.
%Rates of migration estimated from $F_{ST}$ tend to be between 1 and 100 individuals per generation dispersing across all subpopulations.
%
%
%\tmpsection{Measures from range}
%
%The final measurement that has been used is derived from the shape of the species' range, typically from IUCN maps \cite{iucn}.
%
%

\tmpsection{The gap}
\tmpsection{What I did/found}

%#7. Here I use a phylogenetic comparative approach to understand the relationship between pop structure and pathogen richness across the largest study of bats to date. I use a phylogenetic GLM controlling for the other life history characteristics known to impact pathogen richness to quantify the relationship between viral richness (as a proxy for pathogen richness_ and two measures of population structure. 
%#8. I found ...

There is a lack of studies using multiple measures of population structure and larger data sets to robustly estimate the importance of population structure.
Here I have used two measures of population structure --- the number of subspecies and effective gene flow --- to robustly test for an association between population structure and pathogen richness in bats.
Furthermore, I have used a data set that is much larger that previous studies for one of these analyses, further promoting robustness of results.
I found that both measures of population structure are positively associated with viral richness and are included as explanatory variables in the best models for describing viral richness.
Furthermore, I found that the role of phylogeny is very weak in the models and in the distribution of viral richness amongst taxa.


%%%%%%%%%%%%%%%%%%%%%%%%%%%%%%%%%%%%%%%%%%%%%%%%%%%%%%%%%%%%%%%%%%%%%%%%%%%%%%%%%%%%%%%%%%%%%%%%%%%%%%%%%%%%%%%%%%%%%%%%%%%%%%%%%%%%%%%%%%%%%%%%%%%%%%%%%%%

\section{Methods}

%%%%%%%%%%%%%%%%%%%%%%%%%%%%%%%%%%%%%%%%%%%%%%%%%%%%%%%%%%%%%%%%%%%%%%%%%%%%%%%%%%%%%%%%%%%%%%%%%%%%%%%%%%%%%%%%%%%%%%%%%%%%%%%%%%%%%%%%%%%%%%%%%%%%%%%%%%%























































































































